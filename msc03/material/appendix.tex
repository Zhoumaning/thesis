\appendix
\chapter{Kudosten j�rjestys}\label{liiteA}


{\bf Lista kudoksista, joiden osalta geenien ilmentymisdataa oli k�yt�ss�. 21 ensimm�ist� kudosta ovat hiirelle ja ihmiselle yhteisi�. Kudokset on j�rjestetty simuloidulla j��hdytyksell� niin, ett� koko aineistossa samankaltaisesti ilmenev�t kudokset ovat mahdollisimman l�hell� toisiaan. Yhteiset (0-20) ja muut (21-45) kudokset on j�rjestetty erikseen. Muiden kudosten osalta j�rjestyst� ei hy�dynnetty tulosten tulkinnassa.}

\begin{tabular}{r|ll}\toprule
Kudoksen numero  & Ihminen 	& Hiiri \\ \midrule

	0	&Liver	&liver	\\
	1	&Kidney	&kidney	\\
	2	&Adrenal gland	&adrenal gland	\\
	3	&Placenta	&placenta	\\
	4	&DRG	&drg	\\
	5	&Spinal cord	&spinal cord lower	\\
	6	&Cerebellum	&cerebellum	\\
	7	&Cortex	&cortex	\\
	8	&Amygdala	&amygdala	\\
	9	&Caudate nucleus	&striatum	\\
	10	&Testis	&testis	\\
	11	&Trachea	&trachea	\\
	12	&Lung	&lung	\\
	13	&Spleen	&spleen	\\
	14	&Thymus	&thymus	\\
	15	&Ovary	&ovary	\\
	16	&Uterus	&uterus	\\
	17	&Prostate	&prostate 	\\
	18	&Thyroid	&thyroid	\\
	19	&Salivary gland	&salivary gland	\\
	20	&Heart	&heart	\\ \bottomrule
	21	&Pancreas	&eye	\\
	22	&Pituitary gland	&olfactory bulb	\\
	23	&Lymphoblastic molt-4	&frontal cortex	\\
	24	&Myelogenous k-562	&hippocampus	\\
	25	&THY-	&hypothalamus	\\
	26	&THY+	&spinal cord upper	\\
	27	&OVR278E	&trigeminal	\\
	28	&OVR278S	&brown fat	\\
	29	&Prostate Cancer	&skeletal muscle	\\
	30	&Corpus callosum	&tongue	\\
	31	&Thalamus	&epidermis	\\
	32	&Whole brain	&bone	\\
	33	&Fetal brain	&bone marrow	\\
	34	&Bukitts Raji	&umbilical cord	\\
	35	&Burkitts Daudi	&bladder	\\
	36	&K422	&large intestine	\\
	37	&WSU	&small intestine	\\
	38	&Ramos	&stomach	\\
	39	&GA10	&snout epidermis	\\
	40	&DOHH2	&digits	\\
	41	&HL60	&mammary gland	\\
	42	&A2058	&lymph node	\\
	43	&HUVEC	&adipose tissue	\\
	44	&Hep3b	&gall bladder	\\
	45	&Fetal liver	&-	\\ \bottomrule

\end{tabular}





