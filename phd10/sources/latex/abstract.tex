\thispagestyle{empty}
\noindent{{\bf {\Large ABSTRACT}}}
\vspace{0.5cm}

\noindent Lahti, L. (2010): \newblock {\bf Probabilistic analysis of
the human transcriptome with side information} Doctoral thesis, Aalto
University School of Science and Technology, Dissertations in
Information and Computer Science, TKK-ICS-D19, Espoo, Finland.\\%[2mm]

\noindent \textbf{Keywords:} data integration, exploratory data
analysis, functional genomics, probabilistic modeling,
transcriptomics\\

Recent advances in high-throughput measurement technologies and
efficient sharing of biomedical data through community databases have
made it possible to investigate the complete collection of genetic
material, the genome, which encodes the heritable genetic program of an
organism. This has opened up new views to the study of living
organisms with a profound impact on biological research.

Functional genomics is a subdiscipline of molecular biology that
investigates the functional organization of genetic information.  This
thesis develops computational strategies to investigate a key
functional layer of the genome, the transcriptome. The time- and
context-specific transcriptional activity of the genes regulates the
function of living cells through protein synthesis. Efficient
computational techniques are needed in order to extract useful
information from high-dimensional genomic observations that are
associated with high levels of complex variation. Statistical learning
and probabilistic models provide the theoretical framework for
combining statistical evidence across multiple observations and the
wealth of background information in genomic data repositories.

This thesis addresses three key challenges in transcriptome
analysis. First, new preprocessing techniques that utilize side
information in genomic sequence databases and microarray collections
are developed to improve the accuracy of high-throughput microarray
measurements.  Second, a novel exploratory approach is proposed in
order to construct a global view of cell-biological network activation
patterns and functional relatedness between tissues across normal
human body. Information in genomic interaction databases is used to
derive constraints that help to focus the modeling in those parts of
the data that are supported by known or potential interactions between
the genes, and to scale up the analysis. The third contribution is to
develop novel approaches to model dependency between co-occurring
measurement sources. The methods are used to study cancer mechanisms
and transcriptome evolution; integrative analysis of the human
transcriptome and other layers of genomic information allows the
identification of functional mechanisms and interactions that could
not be detected based on the individual measurement sources. Open
source implementations of the key methodological contributions have
been released to facilitate their further adoption by the research
community.

