%\begin{authorscontribution}
% Lisää tähän tekijän osuutta koskeva teksti. / Add here the section concerning author's contribution.

\section*{SUMMARY OF PUBLICATIONS AND THE\\ AUTHOR'S CONTRIBUTION}
\addcontentsline{toc}{section}{Summary of publications and the
  author's contribution} \markboth{THE AUTHOR'S CONTRIBUTION}{THE
  AUTHOR'S CONTRIBUTION}

The publications in this thesis have been a joint effort of all
authors; key contributions by the author of this thesis are summarized
below.

Publication~\ref{PECA} introduces a novel analysis strategy to improve
the accuracy and reproducibility of the measurements in genome-wide
transcriptional profiling studies. A central part of the approach is
the utilization of side information in external genome sequence
databases.  The author participated in the design of the study,
suggested the utilization of external sequence data, implemented this,
as well as participated in preparing the manuscript.

Publication \ref{RPA} provides a probabilistic framework for
probe-level gene expression analysis. The model combines statistical
power across multiple microarray experiments, and is shown to
outperform widely-used preprocessing methods in differential gene
expression analysis. The model provides tools to assess probe
performance, which can potentially help to improve probe and
microarray design. The author had a major role in designing the
study. The author derived the formulation, implemented the model,
performed the probe-level experiments, as well as coordinated the
manuscript preparation. The author prepared an accompanied open source
implementation which has been published in BioConductor, a reviewed
open source repository for computational biology algorithms.

Publication~\ref{NR} introduces a novel approach for organism-wide
modeling of transcriptional activity in genome-wide interaction
networks. The method provides tools to analyze large collections of
genome-wide transcriptional profiling data. The author had a major
role in designing the study. The author implemented the algorithm,
performed the experiments, as well as coordinated the manuscript
preparation. The author participated in and supervised the preparation
of an accompanied open source implementation in BioConductor.

Publication~\ref{MLSP} introduces a regularized dependency modeling
framework with particular applications in cancer genomics. The author
had a major role in formulating the biomedical modeling task, and in
designing the study. The theoretical model was jointly developed by
the author and S. Kaski. The author derived and implemented the model,
carried out the experiments, and coordinated the manuscript
preparation. The author supervised and participated in the preparation
of an accompanied open source implementation in BioConductor.

Publication \ref{ECML} introduces the associative clustering
principle, which is a novel data integration framework for dependency
detection with direct applications in functional genomics. The author
participated in implementation of the method, had the main
responsibility in designing and performing the functional genomics
experiments, as well as participated in preparing the manuscript.

Publication \ref{AC} contains the most extensive treatment of the
associative clustering principle. In addition to presenting detailed
theoretical considerations, this work introduces new sensitivity
analysis of the results, and provides a comprehensive validation in
bioinformatics case studies. The author participated in designing the
experiments, performed the comparative functional genomics experiments
and technical validation, as well as participated in preparing the
manuscript.
